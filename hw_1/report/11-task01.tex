\section*{Задача \textnumero1\\\textit{Предельные теоремы теории вероятностей}}

\subsection*{Условие}
В Москве рождается в год около 122500 детей.
Считая вероятность рождения мальчика равной 0.51,
найти вероятность того, что число мальчиков,
которые родятся в Москве в текущем году,
превысит число родившихся девочек не менее, чем на 1500.


\subsection*{Решение}

Пусть:
\begin{itemize}
    \item X - случайная величина, принимающая значения равные числу мальчиков, родившихся за год в Москве.
    \item Y - случайная величина, принимающая значения равные числу девочек, родившихся за год в Москве.
\end{itemize}

Из условия имеем следующую систему уравнений:
\begin{equation*}
    \begin{cases}
        X + Y = 122500 \\
        X - Y \geqslant 1500
    \end{cases}
\end{equation*}
Следовательно:
\begin{equation*}
    X \geqslant 62000
\end{equation*}

Условие данной задачи описывает схему испытаний Бернулли. При этом $n = 122500$~--- число испытаний. Тогда по интегральной теореме Муавра-Лапласа имеем:
\begin{equation*}
    P\Big\{k_1 \leqslant{} k \leqslant k_2\Big\} = P\Big\{62000 \leqslant{} k \leqslant 122500\Big\} \approx \Phi(x_2) - \Phi(x_1)
\end{equation*}
где, $k$~--- число успехов; $x_i = \frac{k_i - np}{\sqrt{npq}}, i = \overline{1,2}$; $p = 0.51$~--- вероятность успеха, $q = 1 - p = 0.49$~--- вероятность неудачи.
\begin{flalign*}
    &
    P\Big\{x \in [62000, 122500]\Big\} =
    \Phi\bigg(\frac{122500-62475}{\sqrt{30612.75}}\bigg) -
    \Phi\bigg(\frac{6200-62475}{\sqrt{30612.75}}\bigg) \approx\\&
    \approx \Phi(343) + \Phi(2.7) \approx 
    \underline{\underline{0.9964}}
    &
\end{flalign*}

