\section*{Задача \textnumero1\\\textit{Проверка параметрических гипотез}}

\subsection*{Условие}
До наладки станка была проверена точность изготовления $n_1 = 10$ втулок,
в результате чего получено значение $S^2 {(\vec x_{n_1})} = 9.6$ мкм$^2$ . После наладки была проверена
партия из $n_2 = 15$ втулок и получено значение $S^2 {(\vec y_{n_2})} = 5.7$ мкм$^2$ . Считая распределение контролируемого признака нормальным, при уровне значимости $\alpha = 0.05$ проверить гипотезу о том, что после наладки станка точность изготовления втулок увеличилась.

\subsection*{Решение}

Введём основную гипотезу:
\begin{equation*}
    H_0 = \{ \text{точность станка не изменилась} \} = \{ \sigma_1 = \sigma_2\}
\end{equation*}

Введём конкурирующую гипотезу:
\begin{equation*}
    H_1 = \{ \text{после наладки точность увеличилась} \} = \{ \sigma_1 > \sigma_2\}
\end{equation*}

Для рассматриваемого случая воспользуемся статистикой:
\begin{equation*}
    T(\vec x_{n_1}, \vec y_{n_2}) = \frac{max\{{S}^2(\vec x_{n_1}), {S}^2(\vec y_{n_2})\}}
    {min\{{S}^2(\vec x_{n_1}), {S}^2(\vec y_{n_2})\}} \sim F(n_1 - 1, n_2 - 1)
\end{equation*}

Согласно условию
\begin{equation*}
    T(\vec X_{n_1}, \vec Y_{n_2}) \geq F_{1 - \alpha}(n_1 - 1, n_2 - 2)
\end{equation*}

 построим критическую область $W$
\begin{equation*}
    W = \{(\vec x_{n_1}, \vec y_{n_2}) : 
    T(\vec x_{n_1}, \vec y_{n_2}) \geq F_{1 - \alpha}(n_1 - 1, n_2 - 2) \},
\end{equation*}
где $F_{1- \alpha}$ - квантиль распределения Фишера.

Проведём вычисления:
\begin{equation*}
    T(\vec x_{n_1}, \vec y_{n_2}) = \frac{S^2(\vec x_{n_1})}{S^2(\vec y_{n_2})} = \frac{9.6}{5.7} = 1.68
 \end{equation*}

 \begin{equation*}
     F_{1 - \alpha} = F_{0.95}(9, 14) = 2.65
 \end{equation*}

 Вывод:
 \begin{equation*}
     1.68 \ngeq 2.65~ \Rightarrow ~ (\vec x_{n_1}, \vec y_{n_2}) \notin W ~
     \Rightarrow ~ \text{отклоняем гипотезу $H_1$, принимаем $H_0$.}
 \end{equation*}

 \textbf{Ответ:} после наладки станка точность изготовления втулок не увеличилась при уровне значимости $\alpha = 0.05$.
