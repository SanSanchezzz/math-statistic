\section*{Задача \textnumero2\\\textit{Метод моментов}}

\subsection*{Условие}
\sloppy С использованием метода моментов для случайной выборки $\vec{X} = (X_1,\dots,X_n)$ из генеральной совокупности $X$ найти точечные оценки указанных параметров заданного закона распределения:
\begin{equation*}
f_X(x) = \frac{1}{ 2^\frac{\theta{}}{2} \Gamma(\frac{\theta{}}{2}) }x^{\frac{\theta}{2}-1}e^\frac{-x}{2},\quad{}x > 0
\end{equation*}

\subsection*{Решение}

\begin{equation} \label{eq:equation}
    f_X(x) = \frac{1}{ 2^\frac{\theta{}}{2} \Gamma(\frac{\theta{}}{2}) }x^{\frac{\theta}{2}-1}e^\frac{-x}{2} = \frac{\frac{1}{2}^\frac{\theta}{2}}{\Gamma(\frac{\theta}{2})} x^{\frac{\theta}{2} - 1} e^{-\frac{1}{2}x}
\end{equation}
Гамма распределение.

Данный закон распределения зависит от единственного параметра $\theta$, следовательно система метода моментов будет содержать только одно уравнение с одной неизвестной. Это уравнение имеет вид:
\begin{equation} \label{eq:equation1}
    MX = \hat{\mu}_1(\overrightarrow{X})
\end{equation}
При этом:
\begin{equation} \label{eq:equation2}
    \hat{\mu}_1(\overrightarrow{X}) = \overline{X}
\end{equation}

Теперь необходимо вычислить математическое ожидание случайной величины, подчиняющейся данному закону распределения:
\begin{equation*}
    MX = \int^{+\infty}_{-\infty}x \cdot f_X(x)dx
\end{equation*}

\begin{flalign}
    &
        MX =
        \int^{+\infty}_0 x\frac{\frac{1}{2}^\frac{\theta}{2}}{\Gamma(\frac{\theta}{2})}
        x^{\frac{\theta}{2} - 1} e^{-\frac{1}{2}x} dx= 
        \int^{+\infty}_0 \frac{\frac{1}{2}^\frac{\theta}{2}}{\Gamma(\frac{\theta}{2})}
        x^{\frac{\theta}{2} - 1} e^{-\frac{1}{2}x} dx=
        \begin{bmatrix}
            \frac{x}{2} = t\\
            x = 2t
        \end{bmatrix} =
    \nonumber &\\
              & =\int^{+\infty}_0 \frac{(2t)^\frac{\theta}{2} \frac{1}{2}^\frac{\theta}{2} e^{-t}} {\Gamma(\frac{\theta}{2})} 2dt= 
              \frac{2}{\Gamma(\frac{\theta}{2})} \int^{+\infty}_0 t^{\frac{\theta}{2}} e^{-t} dt =
              \frac{2 \Gamma(\frac{\theta}{2} + 1)}{\Gamma(\frac{\theta}{2})} =
              \frac{2 \Gamma(\frac{\theta}{2})\frac{\theta}{2}}{\Gamma(\frac{\theta}{2})} =
              \theta
    \label{eq:mxint} &
\end{flalign}
\\

Приравняем теоретические моменты к их выбранным аналогам, найдём неизвестный параметр:
\begin{equation} \label{eq:mx}
    MX = \overline{x} = \theta
\end{equation}

\textbf{Ответ:} $\theta = \overline{x}$
