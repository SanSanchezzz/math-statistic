\section*{Задача \textnumero4\\\textit{Доверительные интервалы}}

\subsection*{Условие}
\sloppy По результатам n = 10 измерений прибором, не имеющим систематической ошибки, получены следующие отклонения емкости конденсатора от номинального значения (пФ):
\begin{equation*}
    5.4, -13.9, -11.0, 7.2, -15.6, 29.2, 1.4, -0.3, 6.6, -9.9.
\end{equation*}

Найти 90\%-ые доверительные интервалы для среднего значения отклонения емкости от номинального значения и ее среднего квадратичного отклонения.

\subsection*{Решение}

%Пусть $q_{\alpha}$, $q_{1 - \alpha}$~--- квантили соответствующих уровней случайной величины $2S/T$, распределенной по закону $\chi^2$ с степенями свободы $n = 2d = 10$. При этом $2\alpha + \gamma = 1 \Rightarrow \alpha = \frac{1 - \gamma}{2} = \frac{1 - 0.8}{2} = 0.1$. Таким образом:
%\begin{equation*}
    %\gamma = P\Big\{q_{\alpha} < \frac{2S}{T} < q_{1 - \alpha}\Big\}
%\end{equation*}
%Из полученного тождества найдём оценку параметра $T$:
%\begin{equation*}
    %\gamma = P\Big\{\frac{2S}{q_{1 - \alpha}} < T < \frac{2S}{q_{\alpha}}\Big\}
%\end{equation*}

%Значение квантилей $q_{\alpha}$, $q_{1 - \alpha}$ можем вычислить при помощи функции chi2inv пакета MATLAB\@:
%\begin{equation*}
    %q_{0.1} = 4.8652,\quad q_{0.9} = 15.9872
%\end{equation*}

%Имеем итоговую оценку:
%\begin{equation*}
    %\underline{\underline{0.8 \approx P\Big\{200.16 < T < 657.73\Big\}}}
%\end{equation*}


\begin{enumerate}
\item Пусть $X$ -- случайная величина, принимающая значения отклонения ёмкости конденсатора от номинального значения. 

Необходимо построить доверительные интервалы для математического ожидания и среднего квадратичного отклонения.

\item Построим доверительный интервал для математического ожидания $\theta$. 

Так как среднеквадратичное отклонение неизвестно используем центральную статистику. 

$$T(\vec X, \theta)=\frac{\theta-\overline X}{S(\vec X)}\sqrt{n} \sim St(n-1)$$

Следуя общей идее построения доверительных интервалов, выберем $\alpha_1,\alpha_2 > 0$ такие, что $\alpha_1+\alpha_2 = 1- \gamma$. В соответствии со свойствами непрерывных случайных величин можно записать

$$\gamma=P\{t_{\alpha_1}<T(\vec X, \theta)<t_{1-\alpha_2}\}$$

где $t_{\alpha_1},~t_{\alpha_2}$ -- квантили соответствующих уровней распределения Стьюдента с $n-1=9$ степенями свободы. 

Поскольку размах доверительного интервала обычно стараются минимизировать, то выбирают
$\alpha_1=\alpha_2=\frac{1-\gamma}{2}$ , поэтому $t_{1-\alpha_2}=t_{1-\frac{1-\gamma}{2}}=t_{\frac{1+\gamma}{2}}$. В силу симметричности функции плотности
стандартного нормального распределения заключаем, что $t_{\alpha_1}=-t_{\alpha_2}=-t_{\frac{1+\gamma}{2}}$ 

Тогда

$$\gamma=P\bigg\{-t_{\frac{1+\gamma}{2}}<\frac{\theta-\overline X}{S(\vec X)}\sqrt{n}<t_{\frac{1+\gamma}{2}}\bigg\}$$

Или

$$\gamma=P\bigg\{\overline X-\frac{t_{\frac{1+\gamma}{2}}S(\vec X)}{\sqrt{n}}<\theta<\overline X+\frac{t_{\frac{1+\gamma}{2}}S(\vec X)}{\sqrt{n}}\bigg\}$$

Тогда в качестве верхней и нижней границ $\gamma$-доверительного интервала для параметра $\theta$ могут быть использованы статистики:

$$\underline\theta(\vec X)=\overline X-\frac{t_{\frac{1+\gamma}{2}}S(\vec X)}{\sqrt{n}},$$

$$\overline\theta(\vec X)=\overline X+\frac{t_{\frac{1+\gamma}{2}}S(\vec X)}{\sqrt{n}}$$

Вычислим:

$$\overline X=\frac{1}{n}\sum_{i=1}^nX_i=-0.09$$

$$\frac{1+\gamma}{2}=\frac{1+0.9}{2}=0.95$$

$$t_{\frac{1+\gamma}{2}}=t_{0.95}=1.8331$$

\begin{equation*}
S(\vec X)=\frac{1}{n-1}\sum_{i=1}^n(X_i-\overline X)^2=166.78
\end{equation*}

$$\underline\theta(\vec X)=\overline X-\frac{t_{\frac{1+\gamma}{2}}S(\vec X)}{\sqrt{n}}=
-0.09 -\frac{1.8331\cdot 166.78}{\sqrt{10}}=-96.77$$

$$\overline\theta(\vec X)=\overline X+\frac{t_{\frac{1+\gamma}{2}}S(\vec X)}{\sqrt{n}}=
-0.09 +\frac{1.8331\cdot 166.78}{\sqrt{10}}=96.59$$

\item Построим доверительный интервал для среднеквадратичного отклонения $\sigma$. 

Используем центральную статистику. 

$$T(\vec X, \sigma^2)=\frac{S^2(\vec X)}{\sigma^2}(n-1) \sim \chi^2(n-1)$$


Следуя общей идее построения доверительных интервалов, выберем $\alpha_1,\alpha_2 > 0$ такие, что $\alpha_1+\alpha_2 = 1- \gamma$. В соответствии со свойствами непрерывных случайных величин можно записать

$$\gamma=P\{h_{\alpha_1}<T(\vec X, \sigma^2)<h_{1-\alpha_2}\}$$

где $h_{\alpha_1},~h_{\alpha_2}$ -- квантили соответствующих уровней распределения хи-квадрат с $n-1=9$ степенями свободы. 

Поскольку размах доверительного интервала обычно стараются минимизировать, то выбирают
$\alpha_1=\alpha_2=\frac{1-\gamma}{2}$ , поэтому $h_{1-\alpha_2}=h_{1-\frac{1-\gamma}{2}}=h_{\frac{1+\gamma}{2}}$. 

Тогда $h_{\alpha_1}=h_{\frac{1-\gamma}{2}},~h_{\alpha_2}=h_{\frac{1+\gamma}{2}}$, так как график функции плотности не симметричен. 

Тогда

$$\gamma=P\bigg\{h_{\frac{1-\gamma}{2}}<\frac{S^2(\vec X)}{\sigma^2}(n-1)<h_{\frac{1+\gamma}{2}}\bigg\}$$

Или

$$\gamma=P\bigg\{\frac{1}{h_{\frac{1+\gamma}{2}}}<\frac{\sigma^2}{S^2(\vec X)(n-1)}<\frac{1}{h_{\frac{1-\gamma}{2}}}\bigg\}$$

$$\gamma=P\bigg\{\frac{S^2(\vec X)(n-1)}{h_{\frac{1+\gamma}{2}}}<\sigma^2<\frac{S^2(\vec X)(n-1)}{h_{\frac{1-\gamma}{2}}}\bigg\}$$

Тогда в качестве верхней и нижней границ $\gamma$-доверительного интервала для параметра $\sigma^2$ могут быть использованы статистики:

$$\underline\sigma^2(\vec X)=\frac{S^2(\vec X)(n-1)}{h_{\frac{1+\gamma}{2}}},$$

$$\overline\sigma^2(\vec X)=\frac{S^2(\vec X)(n-1)}{h_{\frac{1-\gamma}{2}}}$$

\end{enumerate}

Вычислим:

$$\frac{1+\gamma}{2}=\frac{1+0.9}{2}=0.95$$

$$h_{\frac{1+\gamma}{2}}=h_{0.95}=16.92$$

$$\frac{1-\gamma}{2}=\frac{1+0.9}{2}=0.05$$

$$h_{\frac{1-\gamma}{2}}=h_{0.05}=3.33$$

\begin{equation*}
S(\vec X)=\frac{1}{n-1}\sum_{i=1}^n(X_i-\overline X)^2=166.78
\end{equation*}

$$S^2(\vec X)=27815.57$$

$$\underline\sigma^2(\vec X)=\frac{S^2(\vec X)(n-1)}{h_{\frac{1+\gamma}{2}}}=
\frac{27815.57\cdot 9}{16.92}=14795.52$$

$$\overline\sigma^2(\vec X)=\frac{S^2(\vec X)(n-1)}{h_{\frac{1-\gamma}{2}}}=
\frac{27815.57\cdot 9}{3.33}=75177.22$$

\textbf{Ответ:}

Доверительные интервалы уровня 0.9 для математического ожидания (-96.77, 96.59) и среднего квадратичного отклонения (14795.52, 75177.22).

\end{document}
