\chapter{Теоретическая часть}

Пусть $X$ -- случайная величина, закон распределения которой известен с точностью до вектора $\vec \theta = \big( \theta_1, ..., \theta_r \big)$ неизвестных параметров. Для упрощения рассуждений будем считать, что $r = 1$ и

\begin{equation*}
    \vec \theta = \big( \theta_1 \big) = \big( \theta \big) \in \mathbb{R}^1
\end{equation*}

то есть закон распределения случайной величины $X$ зависит от одного скалярного неизвестного параметра.

Пусть $\vec X$ -- случайная выборка объема $n$ из генеральной совокупности $X$. Тогда $\vec x$ -- любая реализация случайной выборки $\vec X$.

\section{$\gamma$-доверительный интервал}

Интервальной оценкой с коэффициентом доверия $\gamma$ ($\gamma$-доверительной интервальной оценкой) параметра $\theta$ называют пару статистик $\underline{\theta} \big( \vec X \big)$ и $\overline{\theta} \big( \vec X \big)$ таких, что

\begin{equation*}
    P \bigg\{ \underline{\theta} \big( \vec X \big) < \theta < \overline{\theta} \big( \vec X \big) \bigg\} = \gamma
\end{equation*}

Поскольку границы интервала являются случайными величинами, то для различных реализаций случайной выборки $\vec X$ статистики $\underline{\theta}(\vec X), \overline{\theta}(\vec X)$ могут принимать различные значения.

Доверительным интервалом с коэффициентом доверия $\gamma$ ($\gamma$-доверительным интервалом) параметра $\theta$ называют интервал $\big( \underline{\theta} (\vec x),\overline{\theta} (\vec x) \big)$, отвечающий выборочным значениям статистик $\underline{\theta} \big( \vec X \big)$ и $\overline{\theta} \big( \vec X \big)$.

\section{Формулы для вычисления границ \\ $\gamma$-доверительного интервала}

Пусть генеральная совокупность $X$ распределена по нормальному закону с параметрами $\mu$ и $\sigma^2$.

Формулы для вычисления границ $\gamma$- доверительного интервала для математического ожидания:
$$
\underline\mu(\vec X_n)=\overline X - \frac{S(\vec X)t_{\frac{1+\gamma}{2}}}{\sqrt{n}}
$$

$$
\overline\mu(\vec X_n)=\overline X + \frac{S(\vec X)t_{\frac{1+\gamma}{2}}}{\sqrt{n}}
$$

где
\begin{itemize}
    \item $\overline X$ -- точечная оценка математического ожидания;

    \item $S^2(\vec X)$ -- исправленная выборочная дисперсия;

    \item $n$ -- объем выборки;

    \item $\gamma$ -- уровень доверия;

    \item $t_{\frac{1+\gamma}{2}}$ -- квантиль соответствующего уровня распределения Стьюдента с n - 1 степенями свободы.
\end{itemize}

Формулы для вычисления границ $\gamma$- доверительного интервала для дисперсии:

$$
\underline\sigma^2(\vec X_n)= \frac{(n-1)S^2(\vec X)}{h_{\frac{1+\gamma}{2}}}
$$

$$
\overline\sigma^2(\vec X_n)= \frac{(n-1)S^2(\vec X)}{h_{\frac{1-\gamma}{2}}}
$$
где
\begin{itemize}
    \item $S^2(\vec X)$ -- исправленная выборочная дисперсия;

    \item $n$ -- объем выборки;


    %\item $h_{\frac{1+\gamma}{2}}$ -- квантили соответствующих уровней распределения хи-квадрат с n - 1 степенями свободы;
    \item $h_q$ -- квантиль уровня $q \in{\big\{ {\frac{1-\gamma}{2}}; {\frac{1+\gamma}{2}}}\big\}$ распределения хи-квадрат с n - 1 степенями свободы;
    \item $\gamma$ -- уровень доверия.
        
\end{itemize}

